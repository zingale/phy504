\documentclass[10pt]{article}

\usepackage[tagged]{accessibility}

%\usepackage[cm]{fullpage}
\usepackage[lmargin=0.65in,rmargin=0.65in,
            tmargin=0.5in,bmargin=0.75in]{geometry}


\usepackage[parfill]{parskip}

\usepackage{hyperref}

\usepackage{palatino}

% URLs (special font for monospace)

\usepackage[defaultsans]{cantarell} % helvet} %
\usepackage[T1]{fontenc}



\newenvironment{itemsquish}
  { \begin{itemize}
    % set spacing between items
    \addtolength{\itemsep}{-0.25\baselineskip}
    % set spacing between lines
    \addtolength{\baselineskip}{-0.25\baselineskip} }
  { \end{itemize} }


\usepackage{colortbl}

% multicolumn row coloring: http://www.latex-community.org/viewtopic.php?f=5&t=2565
\newcommand{\SetRowColor}[1]{\noalign{\gdef\RowColorName{#1}}\rowcolor{\RowColorName}}
\newcommand{\mymulticolumn}[3]{\multicolumn{#1}{>{\columncolor{\RowColorName}}#2}{#3}}
\definecolor{myGray}{rgb}{0.95,0.95,0.95}

\hypersetup{
  pdftitle={PHY 504: Computational Methods in Physics and Astrophysics I (Spring 2022)},
  pdflang={en-US}
}



\begin{document}

\begin{center}
{\LARGE \sffamily \bfseries PHY 504: Computational Methods in Physics and Astrophysics I} \\[1mm]
{\bfseries Spring 2022} \\
MWF 9:15 am to 10:10am, in Math S235S\\
{\em  Instructor:} Michael Zingale, ESS 452, michael.zingale@stonybrook.edu \\
{\em Web:}\/ \url{https://zingale.github.io/phy504/}
\end{center}



\subsection*{Course Objectives}

\noindent An introduction to procedural and object-oriented
programming in a high-level language such as C++ or modern Fortran
with examples and assignments consisting of rudimentary algorithms for
problems in physics and astronomy. Students will use the UNIX/Linux
operating system to write programs and manage data, and the course
will include an introduction to parallel computing and good
programming practices such as version control and verification. The
course will prepare students for courses in algorithms and methods
that assume a knowledge of programming.

\subsection*{Course Format}

The best way to learn programming is to do it yourself.  The class
meetings will consist of live programming, with students following
along with the instructor.  Some prompts for the day's discussion will
be uploaded to the course website and then updated after class to
cover the details from the day.  Class discussion will take place via
slack in-between meetings.

\subsection*{Office Hours}

\noindent TBA


\subsection*{Grading}

\begin{itemsquish}
\item Homework: 80\%
\item Final Project: 20\%
\end{itemsquish}


\subsection*{Texts}

\noindent There is 1 required text:

\begin{itemize}
\item {\em Introduction to Programming with C++ for Engineers} by Cyganek (IEEE Press)
\end{itemize}

This will be supplemented with online resources.


\subsection*{ Preliminary List of Topics}

\begin{itemize}

\item Unix shell and Bash

\item Editors

\item Version control and git

\item C++ datatypes

\item C++ functions and classes

\item Advanced C++

\item Make

\item Plotting

\item HTML + CSS + web hosting

\item Continuous integration

\item Debugging

\item Profiling

\item Parallel programming

\end{itemize}


\subsection*{Exams}

\noindent There are no exams in this class.

\subsection*{Code Policy}

The assignments and project all involve writing code.  You are
free to use your own machine to develop and test your code, but
{\em \bfseries you are responsible for ensuring that your code builds on the
MathLab machines using g++}.  If I cannot build it using the classroom
machines, you may not get credit.

\subsection*{Final Project}

\noindent All students will do a final project which counts for 20\%
of your grade.  We will discuss these at the midpoint in class.  Some
example projects will be given, and may consist of writing a moderately
complex C++ program.

\subsection*{Homework Policy}

\noindent It is understood that students will discuss the homework
assignments with one-another, however, when writing your solutions,
you must do your own work.  Copying will not be tolerated.



\subsection*{Student Accessibility Support Center Statement}

If you have a physical, psychological, medical, or learning disability
that may impact your course work, please contact the Student
Accessibility Support Center, Stony Brook Union Suite 107, (631)
632-6748, or at sasc@stonybrook.edu. They will determine with you what
accommodations are necessary and appropriate. All information and
documentation is confidential.

Students who require assistance during emergency evacuation are
encouraged to discuss their needs with their professors and the
Student Accessibility Support Center. For procedures and information
go to the following website:
{\small \url{https://ehs.stonybrook.edu//programs/fire-safety/emergency-evacuation/evacuation-guide-disabilities}}
and search Fire Safety and Evacuation and Disabilities.

\subsection*{Academic Integrity}

Each student must pursue his or her academic goals honestly
and be personally accountable for all submitted work. Representing
another person's work as your own is always wrong. Faculty are
required to report any suspected instances of academic dishonesty to
the Academic Judiciary. Faculty in the Health Sciences Center (School
of Health Technology \& Management, Nursing, Social Welfare, Dental
Medicine) and School of Medicine are required to follow their
school-specific procedures. For more comprehensive information on
academic integrity, including categories of academic dishonesty,
please refer to the academic judiciary website at
\url{http://www.stonybrook.edu/commcms/academic\_integrity/}

\subsection*{Critical Incident Management}

Stony Brook University expects students to respect the rights,
privileges, and property of other people. Faculty are required to
report to the Office of Student Conduct and Community Standards any
disruptive behavior that interrupts their ability to teach,
compromises the safety of the learning environment, or inhibits
students' ability to learn. Faculty in the HSC Schools and the School
of Medicine are required to follow their school-specific
procedures. Further information about most academic matters can be
found in the Undergraduate Bulletin, the Undergraduate Class Schedule,
and the Faculty-Employee Handbook.

Until/unless the latest COVID guidance is explicitly amended by SBU,
during Spring 2022 ``disruptive behavior'' will include refusal to wear
a mask during classes.  For the latest COVID guidance, please refer to:
\url{https://www.stonybrook.edu/commcms/strongertogether/latest.php}.


\subsection*{Electronic Communication}

Email to your University email account is an important way
of communicating with you for this course.  For most students the
email address is `{\tt firstname.lastname@stonybrook.edu}'.
%, and the account can be accessed here.
{\em It is your responsibility to read your email received at this
  account.}  For instructions about how to verify your University
email address see this: \\
{\footnotesize \url{http://it.stonybrook.edu/help/kb/checking-or-changing-your-mail-forwarding-address-in-the-epo}}\\
If you choose to forward your University email to another account, we
are not responsible for undeliverable messages.

\subsection*{Religious Observances}

See the policy statement regarding religious holidays at\hfill\\
{\footnotesize \url{http://www.stonybrook.edu/commcms/provost/faculty/handbook/employment/religious_holidays_policy.php}} \\
%
Students are expected to notify the course professors by email of
their intention to take time out for religious observance.  This
should be done as soon as possible but definitely before the end of
the `add/drop' period.  At that time they can discuss with the
instructor(s) how they will be able to make up the work covered.


\end{document}
